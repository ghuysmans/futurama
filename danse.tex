\documentclass[11pt,a4paper,oneside]{article}
\usepackage{hyperref}
\usepackage[T1]{fontenc}
\usepackage[francais]{babel}
\usepackage[utf8]{inputenc}
\usepackage{url}
\usepackage{tikz}
\usepackage{amsmath}

\begin{document}
\emph{Le problème décrit ici dérive d'Advent of Code
\footnote{\url{https://adventofcode.com/2017/day/16}},
un calendrier de l'Avent avec
des problèmes algorithmiques au lieu de chocolats.}

\section{Introduction}
Une société de petits personnages adore la danse. Chaque année, pour Noël, ils
organisent une grande fête à laquelle leur Grand Roi Vénéré participe. Sa
Majesté n'autorise pas tout et n'importe quoi : seules les danses qui La font
commencer et finir première sont permises. Si elle finit à une autre place, le
chorégraphe est exécuté. C'est ce qui est arrivé au chorégraphe de la Cour
l'année passée : il faut dire qu'avec son grand âge, il n'avait déjà plus toute
sa tête. Cette année, c'est à son fils de prendre la relève : tous les autres
ont bien trop peur de mourir pour un pas de travers. Il doit faire quelque
chose de grandiose s'il veut faire oublier l'erreur de son père et profiter de
ses cadeaux. Le problème, c'est qu'il a laissé tomber ses papiers quelques
jours avant Noël et qu'il avait oublié de les numéroter. Il hésite entre deux
possibilités et il n'a plus le temps de les essayer. Qu'est-ce qu'on attend ?

Ils commencent dans l'ordre ($A\dots P$) et enchaînent trois types de pas :
\begin{itemize}
\item \texttt{spin} $n$ ($\forall n \in \{1\dots15\}$) :
	les $n$ derniers deviennent les $n$ premiers.
\item \texttt{exchange} $i$ $j$ ($\forall i,j \in \{0\dots15\}$) :
	ceux en $i$ et $j$ échangent leur place.
\item \texttt{partner} $ab$ ($\forall a,b \in \{A\dots P\}$) :
	$a$ et $b$ échangent leur place.
\end{itemize}

\section{Positions}
Pour simplifier nos exemples, nous commencerons avec quatre couleurs (pour
quatre personnages) et deux types de pas : \texttt{spin} et \texttt{exchange}.

Par groupes, travaillez sur ces exercices :
\begin{enumerate}
\item Représentez les étapes successives de cette danse :
	\begin{itemize}
	\item (début)
	\item \texttt{spin 3}
	\item \texttt{spin 3}
	\item \texttt{exchange 0 1}
	\item \texttt{exchange 2 3}
	\item \texttt{spin 1}
	\item \texttt{exchange 1 3}
	\end{itemize}
\item \emph{Implémentez} en Pascal ces pas (faites des schémas !) :
	\begin{itemize}
	\item \texttt{procedure spin(n: 1..15);}
	\item \texttt{procedure exchange(i, j: 0..15);}
	\end{itemize}
\item L'ordre est-il important ? Toujours, jamais, parfois ?
\end{enumerate}

\section{Personnages}
Jusqu'ici, nous avons ignoré \texttt{partner} qui échange deux personnages. On
peut choisir de les faire directement ou séparément des autres pas, puisqu'ils
ne dépendent pas du tout des positions !

Par groupes, travaillez sur ces exercices :
\begin{enumerate}
\item De quoi la \texttt{procedure partner(a, b: 'A'..'P');} a-t-elle besoin ?
	Écrivez sa signature (son prototype).
\item Implémentez-les en même temps (deux personnes par sous-programme).
\end{enumerate}

\section{Optimisations}
Ça y est, notre ami a survécu. Il vous sera \emph{éternellement reconnaissant}
mais il a encore un <<~petit quelque chose~>> à nous demander.
%FIXME ça correspond à la traduction

Sa~Majesté se lasse de tout ça et a décidé de changer les règles :
pendant le prochain milliard d'années (ils vivent tous très longtemps mais la
principale cause de mortalité est la décapitation), l'Élu
(qu'Elle essaiera de ne pas exécuter entre-temps, Elle fera un effort)
participera à sa place et ils reprendront du même endroit l'année suivante.
%Grâce à ses talents de négociateur, il a pu obtenir de Sa~Majesté qu'elle
%accepte plus de danses originales (Elle n'est plus toujours première à la fin)
%à condition qu'après un milliard d'années, Elle finisse à nouveau la première.
La danse ne changera pas (quel intérêt si Elle ne regarde pas ?).
Le chorégraphe doit maintenant prédire la position exacte des danseurs après
tout ce temps. S'il se trompe, il sera exécuté, lui et toute sa famille.
%Son règne n'est pas près de s'arrêter : Sa~Majesté est immortelle.

%Exercice : combien de temps nous faudrait-il pour calculer ça naïvement ?

\subsection{Sur un cercle}
Imaginons maintenant que les danseurs se trouvent sur un cercle.

Par groupes, répondez à ces questions en partant d'un schéma :
\begin{enumerate}
\item Où se trouve le premier danseur ?
\item Que fait \texttt{spin 2} ? Est-ce que ça va plus vite ?
\item Que fait \texttt{exchange 0 2} ?
\item Un tableau est-il encore adapté ?
\end{enumerate}

\subsection{Profiling}
Un \emph{profiler} permet de connaître la proportion de temps passé dans
une partie d'un programme. Voici ce qu'il a fourni comme résultat :
%bluff complet
\texttt{partner} passe trop de temps à chercher où se trouvent les caractères
qu'elle doit intervertir (et ce, même si on l'a bien implémentée).

Changeons totalement notre manière de voir les choses : puisqu'on peut faire
séparément ces échanges, on peut utiliser un tableau indicé
par un personnage (\texttt{subst: array['A'..'P'] of 'A'..'P'}) afin d'associer
à chaque personnage celui avec lequel il a été échangé.
Il ne stocke plus où se trouvent les personnages mais plutôt comment les échanger.

Cette définition peut paraître étrange : habituellement, on indice un tableau
avec un entier, pas un caractère. En réalité, on peut voir un caractère
comme un entier sur un octet\footnote{Ce n'est pas vrai en général ($\pi$...)
mais nous n'utilisons que les lettres de A à P. Un~octet (\emph{byte}) est
composé de 8~bits (0 ou 1), ce qui fait $2^8=256$ possibilités.} : en ASCII,
les majuscules commencent à 65 et se suivent, donc B=66, C=67, etc. Il peut
alors retirer 65 de chaque lettre pour obtenir des indices dans $\{0..15\}$ :

\begin{table}[h]
\center
\begin{tabular}{c|c|c}
Lettre & ASCII & Indice \\ \hline\hline
A & 65 & 0 \\ \hline
B & 66 & 1 \\ \hline
C & 67 & 2 \\ \hline
\dots & \dots & \dots \\ \hline
P & 80 & 15
\end{tabular}
\caption{Équivalence d'indices}
\end{table}

Par groupes, travaillez sur ces exercices :
\begin{enumerate}
\item Que contient le tableau avant le premier \texttt{partner xx} ?
\item Construisez un tableau qui représente \texttt{partner AB}.
\item Construisez un autre tableau représentant \texttt{partner BC}.
\item Comment pourriez-vous les \emph{composer} ? L'ordre importe-t-il ?
	Comparez en effectuant manuellement ces transformations sur \texttt{ABCD}.
%\item Comment le tableau transforme-t-il la chaîne obtenue précédemment ?
	%notion d'inverse, p-ê un peu hardcore ici haha
\end{enumerate}

Le tableau qu'on construit peut aussi être vu comme une fonction qui prend comme
paramètre un personnage, et cette fonction change au cours de l'exécution
du programme.

TODO incruster un exemple graphique puis son résumé

\subsection{Un peu de théorie des groupes}
%FIXME quand même vérifier qu'on parle vraiment de groupe mais normalement oui
Les applications ($f:X\rightarrow X$) munies de la composition forment un
groupe\footnote{Voir le cours de maths de 1ère secondaire...} non commutatif
(preuves laissées au lecteur\footnote{Il faut repartir de la définition de
fonction, pas encore vue en 3ème secondaire.}) :
%TODO peut-être lourdingue donc pas dit comme ça, vérifier
\begin{itemize}
\item l'élément neutre est la fonction identité :
	$\forall f \quad (id\circ f) = (f\circ id) = f$
\item la composition de deux fonctions reste une fonction
	\footnote{On dit que la composition est interne.}
\item la composition est associative :
	$\forall f, g, h \quad (f\circ g)\circ h = f\circ (g\circ h)$
%TODO inverse, histoire d'être complet
\end{itemize}

On peut ainsi calculer efficacement la $n$ème puissance d'une application~$f$,
à l'aide de l'algorithme de l'exponentiation rapide qui a rendu possible la
cryptographie moderne qui utilisait de très grands nombres à l'exposant : \[
	f^n = \left\{\begin{array}{ll}
		id &
			\text{si $n=0$} \\
		\left(f^2\right)^{\frac n2} = \left(f\circ f\right)^{\frac n2} &
			\text{si $n$ est pair} \\
		f\circ \left(f\circ f\right)^{\left\lfloor\frac n2\right\rfloor} =
		\left(f\circ f\right)^{\left\lfloor\frac n2\right\rfloor}\circ f &
			\text{si $n$ est impair (par assoc.)}
	\end{array}\right.
\]

%on calcule donc f(g(h(x))) en calculant d'abord (f\circ g\circ h), plus rapide
%TODO parler de l'inverse et de ce qu'on calculait dans le désordre...

\end{document}
